\documentclass[12pt]{article}
\usepackage{graphicx}
\usepackage{fancyvrb}
\usepackage{amsmath}
\usepackage{amssymb}
\usepackage{tikz}
\usepackage[margin=0in]{geometry}
\newtheorem{definition}{Definition}

\title{CS 422 - Quiz 2}
\author{Otavio Sartorelli de Toledo Piza}
\date{Spring 2023}

\begin{document}

\maketitle

\pagebreak

\paragraph*{Quiz Instructions}
I have read the instructions. PUID: 0032690213.

\pagebreak

\paragraph*{Question 1}
Cumulative acknowledgement is when the receiving host on a TCP connection waits to acknowledge
more than one packet at the same time. It helps reduce retransmissions by allowing the host that
sent the packets to only retransmit those that were not acknowledged.

\paragraph*{Question 2}
The sliding window provides a flow control mechanism in which the receiving host advertizes a
window size to the sending host. Then, the sending host will take into account un-acknowledged
packets (i.e. ones that are still in transit) and only send so much new data to the receiving
host as to not overwhelm it. This maximizes the amount of data being transmitted without
the a server overwhelming a client's resources. The sliding window also allows multiple packets
to be in transit at the same time since they contain enough information to be ordered independent
of when they arrive.

\paragraph*{Question 3}
The three way handshake is the initialization sequence that starts a TCP connection. It ensures
that both the sender and receiver are ready to communicate. In terms of performance, it is the
contains the shortest number of steps to initialize a connection that guarantees all packets
will be delivered and will be in order from an application's viewpoint. However, it still adds
some overhead compared to UDP where such a handshake is not present, but it does not guarantee
all packets will be delivered.

\subparagraph*{}
In that sense, it is impossible to build a reliable communication protocol without some form of
handshake since it establishes necessary parameters that allow for guaranteed packet delivery
and ensures both hosts are ready to communicate. For example, it would not be possible to
synchronize both hosts to guarantee they are both receiving all data and interpreting it
correctly.

\paragraph*{Question 4}
Flow control is when senders limit how much data they send to not overwhelm the receiver. Congestion
control is preventing too much data being inserted into a network. In that sense, networks can
become congested in many ways. For example, if there are too many devices the network can become
congested. This can also happen if there is a link that is heavily used and becomes a bottleneck
(e.g. a bridge in terms of a graph).

\pagebreak

\paragraph*{Question 5}
The data plane is where the actual data transferring takes place i.e. the actual physical components
of the network. On the other hand, the control plane is responsible for making decision on how the
data should be forwarded. For example, restricting traffic within a sing VLAN is a control plane's
job. With that in mind, a SDN the control plane is separated from the data plane by introducing
centralized controllers and network hardware that only worries about the data plane which are managed
by the controllers.

\paragraph*{Question 6}
One application is VLAN which ensures traffic only stays within the same VLAN. For example, a host
on VLAN 1 cannot communicate with a host on VLAN 2. More generally, we can say that one of the control
plane's application is to control the access of a group or peer have to the network's resources.
Another application is for routing protocols which determine the best path for traffic. In that sense, 
the centralized controller enables routers to get information about the topology and state of the
network and thus make better decision by using protocols such as OSPF which require a global
network map.

\pagebreak

\paragraph*{Question 7}
A forwarding table is used to store information about the next hop for a given destination address.
For example, a forwarding table would store that for a given address A the next hop is to go to
B. In that sense, the forwarding table has specific details on how to forward a packet to a given
address. On the other hand, the routing table sores information about the topology of the network and
the paths packets have to reach their destination. In that sense, the routing table gets populated
by BGP and allows the router to make forwarding decisions.

\subparagraph*{}
With that in mind, it is clear that the router needs both tables because they each have a different
application. Without the forwarding table, the router would not know how to forward a packet to
address A and without the routing table, the router would not know how to reach a destination since
it would not know of any available routes to get to that destination.

\paragraph*{Question 8}
The key difference is that while distance-vector protocol keeps track about the number of hops (i.e.
distance) and direction to all other networks by exchanging information with its neighbors the
link-state protocol involves each router exchanging information about its connections (i.e. who
it can reach) with all other routers in the entire network and then constructing a
map of the network used to determine the best path to each destination.

\subparagraph*{}
Split horizon with poison reverse is a method for distance-vector protocols which prevent infinite
routing loops. In that sense, in split horizon, a router will not advertize a route back to the
router that gave it that information. For example, if A tells B about C, B will not tell A about C.
Moreover, with poison reverse when a router sees a destination is unreachable it immediately tells
that to the neighbor that originally advertized the route. Using the same example, if B sees that
C is now unreachable, it immediately tells A about that so it avoids forwarding packets towards
that unreachable destination.

\end{document}
