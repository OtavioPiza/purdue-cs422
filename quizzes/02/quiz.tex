\documentclass[12pt]{article}
\usepackage{graphicx}
\usepackage{fancyvrb}
\usepackage{amsmath}
\usepackage{amssymb}
\usepackage{tikz}
\usepackage[margin=0in]{geometry}
\newtheorem{definition}{Definition}

\title{CS 422 - Quiz 2}
\author{Otavio Sartorelli de Toledo Piza}
\date{Spring 2023}

\begin{document}

\maketitle

\pagebreak

\paragraph*{Quiz Instructions}
I have read the instructions. PUID: 0032690213.

\pagebreak
Blank

\pagebreak
Blank

\pagebreak
Blank
\pagebreak

\begin{verbatim}
    











\end{verbatim}

% What's the difference between bandwidth and throughput? To fetch a data of 200
% megabytes over a 10 Gbps link with a round-trip-time (RTT) of 200ms, what'd be the bandwidth
% and throughput of such communication?
\paragraph*{Question 1}
Bandwidth is the maximum number of bits a network can transfer for a unit of time; for example,
each second a given network might be theoretically able to transfer 1 Mb. On the other hand, 
throughput is a measure of how the network performs when transferring data in a given time 
interval. For example, if it takes 2 seconds to transfer a 1 Mb file using the network previously
mentioned from host A to host B, its throughput is 0.5 Mbps. With that in mind,
the bandwidth of the network represents the maximum data that can be transmitted while the throughput
represents actual performance and can be much lower due to factors such as congestions and other
external factors.

\subparagraph*{}
In that sense, the bandwidth of the network mentioned in the question is 10 Gbps since it is
the theoretical maximum performance of that network link. On the other hand, to calculate
its throughput we need to consider the round-trip-time $RTT$ like the following:
\begin{equation*}
    \text{throughput } = \frac{\text{size of the data}}{RTT} = \frac{200 \text{ MB}}{200 \text{ ms}}
                       = 1 \text{ MB/ms} = 1000 \text{ MBps} = 8000 \text{ Mbps}
\end{equation*}
Assuming, Gpbs means gigabits per second, where:
\begin{itemize}
    \item[] MBps = megabytes per second.
    \item[] Mbps = megabits per second.
\end{itemize}

% What’s the difference between latency and jitter? Will jitter have any effect on the
% bandwidth or average throughout (or both) of a communication link or message transfer?
\paragraph*{Question 2}
Latency is a measure of the time it takes to travel from one end of the network to the other for
example a packet sent from host A might take 1 second to reach host B. On the other hand, jitter is
a measure of how consistent, or rather inconsistent, the times of arrival of packages are. For
example, on average packets from A to B might take 1 second but if, for example, a bunch of packets
take 5 seconds and then another bunch arrives in a very short burst we say that there is jitter
in the communication between A and B.

\paragraph*{Question 12}*{}
That said, bandwidth is a theoretical maximum transfer capacity of a network so jitter does not
affect it (it might be a symptom of a congested network). Moreover, jitter just represents the
inconsistentness of packet arrival delay and, thus, on a large  enough average does not affect
throughout (latency does) since there are packets that arrive earlier and packet that arrive later
the the average.

\pagebreak

\begin{verbatim}
    














\end{verbatim}

% What’s an Ethernet LAN segment, and what role do learning bridges play in
% connecting a network of such segments
\paragraph*{Question 3}
A LAN segment is a physical cable that connects hosts on the same cable with the Ethernet layer
using CSMA/CD. A learning bridge also operates on the ethernet layer and can connect different
LAN segments which eliminates the length restriction associated with having multiple hosts talking
over the same physical wire. Moreover, a learning bridge automatically learns which LAN segment a
host is on. Then if A and B are on the same segment and A sends a packet to B the learning bridge
will not broadcast the packet A sent. However, if A sends a packet to C and they are not in the
same segment, the bridge will broadcast the packet to everyone except to the ingress port.

% Synchronizing clocks between nodes is one of the main challenges that any
% encoding scheme must handle carefully. List three such schemes and argue why one is better
% than the other.
\paragraph*{Question 4}
Three such schemes are Non-return to Zero (NRZ), Manchester encoding (Manchester), and Non-return
to Zero inverted (NRZI). One advantage of Non-return to Zero inverted is that different than the Manchester
it does not required the network signal rate to be twice of the bit rate; this alone means that for
the same hardware, NRZI already has twice the bandwidth of Manchester. Moreover, NRZI problem of
having long sequences of 0s and 1s can be improved by using extra bits during encoding that allow it
to be more compact than NRZ. Therefore, I argue that NRZI is the best option among the three provided.

\pagebreak

% Highlight the differences between connectionless datagram, connection-oriented
% switching, and source routing? When is one better than the other? Also, which of these switching
% schemes does the Internet (with the big I) implements?
\paragraph*{Question 5}
Connectionless datagram is sued in packet-switched network such as the Internet: each packet is
an independent unit and is directed based on its destination address without establishing a connection
beforehand (meaning delivery is not guaranteed). This switching approach is efficient for sending
relatively small amounts of data where having packets on the first try is not required. Next,
connection oriented switching a connection is established beforehand which ensures reliability
since that connection is maintained throughout the transmission. In that sense, this type of
switching is ideal for applications such as telephones since there is a smaller chance of jitter,
given the connection route wont change, or packet drops since the connection is already established.
Finally, source routing requires the sender to specify a list of intermediaries that the packet
will pass through to reach its host. While this requires the sender to process a valid path, it also
allows it to choose the route which might allow it to avoid congested nodes or go through insecure
paths: e.g. a classified packet from a Government agency might not want to go through enemy nations.

\subparagraph*{}
The Internet uses connectionless datagram since it is best suited for a huge network that is
constantly changing and where having to establish connections before transmitting, most of the time,
small amounts of data such as a landing page would lead to congestions or having sender compute valid
paths would be computationally expensive. On the other hand, connection-oriented switching is ideal
for applications such as telephones since it grants users with a more reliable and less jittery
connection which connectionless datagram is not able to afford since packets might take different routes
and more likely be lost. Finally source routing is best suited for specialty applications where having
control over how packets are transmitted is paramount, for example a country's government might not
want to have its classified packets go through unfriendly nations who might try to spy on them. In that
sense, source control provides them with that kind of control.

% What’s the difference between (learning) switches, and routers? Which one of these
% requires offline (out of band) support to maintain its forwarding tables?
\paragraph*{Question 6}
Learning switches operate on the ethernet layer (layer 2) while routers operate on the IP layer
(layer 3). In that sense, switches typically use algorithms to populate their tables and do not
require offline support (just plug and play). On the other hand, routers are subject to the topology
of the network and therefore might need offline algorithms to maintain their tables.

\pagebreak

% What’s a spanning tree protocol? What purpose does it serve in a network?
\paragraph*{Question 7}
The spanning tree is a loop free view of the network of a network. Its purpose it to provide a loop
free path for packets to go from any host to any other host since, if a packet entered an infinite
or very long loop, the network could become easily congested. In that sense, the tree structured
computed on top of the real network guarantees the a path between A and B not only exists but also
has not loops which makes sending the packet much simpler.

% What’s the relationship between ARP and broadcast? Moreover, fragmentation
% causes packets to split into smaller packets as they traverse the network; do ARP packets ever
% get fragmented, if not, why?
\paragraph*{Question 8}
ARP packets are themselves broadcasts: since the sender does not know who the receiver is, it has
to broadcast a message to everyone on a network segment requesting the MAC address of a specified
device.

\subparagraph*{}
Fragmentation occurs only when packets are too large to be transmitted over the network. In that
sense, because ARP packets are very small they will not be split into multiple ones.

\pagebreak

% What do we desire from a network addressing scheme? Why is Ethernet so poor at
% meeting this desire; give 2 examples where Ethernet fails?
\paragraph*{Question 9}
The main characteristics of a network addressing schemes is that is has to be globally unique,
scalable, simple to manage, and efficient. While MAC addresses are unique and easy to manage
with each network device manufacturer being provided with a MAC range, they are not scalable
nor efficient. Regarding efficiency, MAC addresses have no structure so even if two devices
only have the LSB different they might be in totally different networks. Therefore, routers would
have to keeps a table of every ethernet address which requires untractable amounts of memory. In
terms of scalability, using MAC addresses could lead to huge broadcast waves from ARP requests
going through the entire Internet trying to find one device which would lead to congestions
rendering the network more unusable as devices are added.

% What’s the difference between classful and classless addressing? How is subnetting
% related to these addressing schemes?
\paragraph*{Question 10}
They are each a different approach of assigning ips. While classful addressing is based on pre-established
templates where a known amount of bits are associated with a network, subnetwork, or hosts, classless
has to store that information somewhere else since the format can vary from packet to packet which means
that information has to be stored in the packet itself. Subnetting is used in both of these schemes to
divide a network into a smaller subnetworking the main difference is that for classful addressing
the number and which bytes are associated with the network, subnetwork, and host are know in advance
just by looking at the first bits of the ip, but for classless addressing the router has to look inside
the ip packet to look for a mask that allows it to extract the bits pertinent to the current network
level.

\pagebreak

% List all the steps involved in performing a `ping` between hosts H1 and H3? (Please
% try to be as detailed as possible.)
\paragraph*{Question 11}
Assuming H1's ARP table is empty, it first makes an ARP request for H3's IP; H2 drops the packet
and R1 changes the MAC address in the request to be its on matches the prefix of H3's IP to that
of the lower (in the picture) subnetwork with the mask 255.255.255.0 and number 128.96.33.0 and
broadcasts the modified ARP request to that network which is answered by H3 with its mac address.
The router than sends that response with H3 eth address to H1. Now, H1 sends a ping request (ICMP)
to H3's MAC and IP, the router read's H3's ip from the ping request and masks it
with the subnet mask 255.255.255.0 and which matches with the lower subnetwork. Then it sends the
packet to that which is received by H3. H3 the sends a response to the packet's source ip which
is received by the router and masked with 255.255.255.255.128 which matches with the upper subnetwork.
The router then forwards the packet to H1 in that network.

% List all the steps involved in performing a `ping` between router R1 and host H3?
% (Please try to be as detailed as possible.) 
\paragraph*{Question 12}
First H3 will make an ARP request for the IP 128.96.33.1 which is answered by the router R1 with its
MAC address and stored on H3's arp table. Then, H3, will make a ping request with H3's IP and ETH
address. This request will be received by R1 which will apply the mask in the request (255.255.255.0)
to the provided ip (128.96.33.1) and see that it matches with the 'current' subnetwork. Then it will
use that mask to get the device (0.0.0.1) and notice that it is itself the device. Then the router
will answer the ping request using the source ip in the packet (128.96.33.14).

\end{document}
